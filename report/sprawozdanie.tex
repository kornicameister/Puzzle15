% vim:encoding=utf8 ft=tex sts=2 sw=2 et:

\documentclass{classrep}
\usepackage[utf8x]{inputenc}
\usepackage[T1]{fontenc}
\usepackage[polish]{babel}
\usepackage{polski}

\studycycle{Informatyka, studia stacjonarne, inż. I st.}
\coursesemester{VI}

\coursename{SISE}
\courseyear{2012/2012}

\courseteacher{mgr inż. Jagoda Lazarek}
\coursegroup{środa, 8:15}

\author{
  \studentinfo{Tomasz Trębski}{158065} \and
  \studentinfo{Sebastian Jakowski}{165415} \and
  \studentinfo{Andrzej Prokopczyk}{165502}
}

\title{Zadanie 1: Piętnastka}

\begin{document}
	\maketitle
	\section{Cel}
		Celem zadania była implementacja dwóch programów:
		\begin{itemize}
		\item rozwiązującego układankę "piętnastka"
		\item wizualizującego kolejne kroki rozwiązania uzyskanego w pierwszym programie
		\end{itemize}
	
	\section{Strategie przeszukiwania}
		W naszym programie zgodnie z wymaganiami wykorzystaliśmy następujące metody przeszukiwania:
		\begin{enumerate}
			\item \textbf{DFS}: przeszukiwanie w głąb polega na badaniu wszystkich krawędzi wychodzących z podanego wierzchołka. Po zbadaniu wszystkich krawędzi wychodzących z danego wierzchołka algorytm powraca do wierzchołka, z którego dany wierzchołek został odwiedzony.
			Kroki algorytmu:
			\begin{enumerate}
				\item Oznacz aktualny węzeł jako odwiedzony
				\item Dla każdego sąsiada aktualnego węzła:
				\begin{enumerate}
					\item Sprawdź czy był on już odwiedzony
					\item Jeśli nie był, rozpocznij procedurę DFS dla tego węzła
				\end{enumerate}
			\end{enumerate}
			\item \textbf{IDFS}: Polega na zbadaniu wszystkich węzłów w odległości d od stanu początkowego, w przypadku braku rozwiązania głębokość przeszukiwania jest zwiększana i całe przeszukiwanie rozpoczyna się od początku.
			Algorytm:
			\begin{enumerate}
				\item Przeszukaj wszystkie węzły w odległości d od stanu początkowego
				\begin{enumerate}
					\item Jeśli stan końcowy został osiągnięty - zakończ
				\end{enumerate}
				\item W przeciwnym wypadku zwiększ głębokość przeszukiwania i rozpocznij procedurę od nowa
			\end{enumerate}
			\item \textbf{BFS}: przeszukiwania grafu wszerz rozpoczyna się od zadanego wierzchołka \textit{s},
			a kończy w momencie osiągnięcia grafu końcowego \textbf{k}. Począwszy od wierzchołka
			początkowego odwiedzane są wszystkie wierzchołki, które można osiągnąć z danego wierzchołka.\\
			Algorytm:
			\begin{enumerate}
				\item Dodaj wierzchołek startowy do kolejki.
				\item Pobierz element z kolejki.
				\begin{itemize}
					\item Jeśli element z kolejki równy szukanemu, koniec algorytmu.
					\item W innym wypadku pobierz wszystkich możliwych sąsiadów i wprowadź ich do kolejki.
					\item Jeśli któryś z sąsiadów jest elementem końcowym, koniec algorytmu. 
				\end{itemize}
				\item Jeśli kolejka pusta, wszystkie węzły w grafie zostały odwiedzone, koniec algorytmu.
				\item W innym wypadku, powrót do kroku 2.
			\end{enumerate}
			Przygotowany przez nasz algorytm BFS, praktycznie zawsze kończy się w wyniku znalezienia
			w jednym z następników danego węzła, węzła końcowego. Dlatego też, krok polegający na 
			sprawdzeniu rozmiaru kolejki nie jest nigdy osiągalny.
			\item \textbf{A-Star}: jest kolejnym z algorytmów szukających ścieżki w grafie, z tą
			różnicą, że chodzi zawsze o najkrótszą ścieżkę od węzła wejściowego do końcowego. \textbf{A*}
			zaczyna od sprawdzenia węzłów, które są sąsiadami danego węzła i które nie zostały jeszcze odwiedzone.
			Jego dużą zaletą jest to, że nie posiada on cech algorytmu zachłannego, ponieważ przy
			wyborze dalszej drogi uwzględnia on koszt dotarcia do następnego węzła, jak i do węzła końcowego.
			Warto wprowadzić więc 3 parametry krawędzi łączącej dwa węzły:
			\begin{itemize}
				\item \textbf{G} - koszt ścieżki prowadzącej z punktu startu do aktualnej, rozpatrywanej pozycji
				w przestrzeni. Innymi słowy jest to całkowity koszt już wygenerowanej ścieżki;
				\item \textbf{H} - wyliczona na podstawie przyjętej heurystyki, szacunkowy koszt dotarcia do
				kolejnego węzła lub do węzła końcowego;
				\item \textbf{F} - całkowity koszt danej krawędzi
			\end{itemize}
			Ostatecznie, aby algorytm poprawnie realizował swoje zadanie, należy zdefiniować 3 kolekcje:
			\begin{itemize}
				\item \textbf{openSet} - zbiór węzłów, do sprawdzania;
				\item \textbf{closedSet} - zbiór węzłów, które już odwiedziliśmy i nie będą one ponownie sprawdzane;
				\item \textbf{priorityQuere} - kolejka węzłów, posortowanych malejąco względem całkowitego kosztu;
			\end{itemize}
		\end{enumerate}
		
	\section{Instrukcja obsługi programu}
		\subsection{Parametry wejściowe}
			Program obsługiwany jest z konsoli i wspiera definiowanie rodzaju strategii przeszukania grafu,
			wyboru heurystyki oraz metody wprowadzenia danych:
			\begin{itemize}
				\item \textbf{Algorytmy przeszukania}:
				\begin{enumerate}
					\item \textbf{-b} dla BFS
					\item \textbf{-d} dla DFS
					\item \textbf{-i} dla iDFS
					\item \textbf{-a} dla A*
				\end{enumerate}
				\item \textbf{Argumenty dla strategii}:
				\begin{enumerate}
					\item dla strategii \textbf{BFS,DFS,IDFS} należy zdefiniować porządek generowania sąsiadów
					\begin{itemize}
						\item \textbf{R} dla losowego porządku
						\item \textbf{LPGD} w dowolnej permutacji - stały porządek
					\end{itemize}
					\item dla strategii \textbf{A*} należy wybrać jedną z dwóch dostępnych heurystyk:
					\begin{itemize}
						\item 1 dla \textbf{ManhattanDistance}
						\item 2 dla \textbf{InvalidCount}
					\end{itemize}
				\end{enumerate}
				\item \textbf{Wprowadzanie danych}:
				\begin{itemize}
					\item \textbf{-f cmd} - stan wejściowy wprowadzany z konsoli
					\item \textbf{-f file=\{plik z wejściowym węzłem\}}
				\end{itemize}
			\end{itemize}
		\subsection{Przykładowe użycia}
			\begin{itemize}
				\item \textbf{-a 1 -f cmd} - A* z ManhattanDistance, węzeł startowy z konsoli,
				\item \textbf{-i R -f file=./input.txt} - iDFS z losowym porządkiem, węzeł startowy ładowany z pliku
			\end{itemize}
	
	\section{Krótki opis zaimplementowanych klas}
		Klasy programu rozwiązującego układankę:
		\begin{enumerate}
			\item IDFSStrategy - klasa implementująca rozwiązanie zagadki przy użyciu algorytmu iDFS
			\item DFSStrategy - klasa implementująca rozwiązanie zagadki przy użyciu algorytmu DFS
			\item AStarPuzzleStrategy - klasa implementująca rozwiązanie zagadki przy użyciu algorytmu A*
			\item BFSPuzzleStrategy - klasa implementująca rozwiązanie zagadki przy użyciu algorytmu BFS
		\end{enumerate}
		Klasy programu wizualizującego uzyskane rozwiązanie
		\begin{enumerate}
			\item
		\end{enumerate}
		
		\section{Część badawcza}
		Statystyki przeszukiwania metodą iDFS dla układanek z rozwiązaniem w odległościach 0-7
		\newline
		Kolejnosc przeszukiwania:LPGD
		Odleglosc rozwiazania: 1
		\begin{itemize}
		\item Średnia dlugosc rozwiazania: 1.0
		\item Średnia ilosc odwiedzonych stanow: 4.0
		\item Średnia maksymalna glebokosc rekursji: 2.0
		\item Średnia ilosc wygenerowanych stanow: 6.5
		\end{itemize}
		Odleglosc rozwiazania: 2
		\begin{itemize}
		\item Średnia dlugosc rozwiazania: 2.0
		\item Średnia ilosc odwiedzonych stanow: 15.75
		\item Średnia maksymalna glebokosc rekursji: 3.0
		\item Średnia ilosc wygenerowanych stanow: 15.0
		\end{itemize}
		Odleglosc rozwiazania: 3
		\begin{itemize}
		\item Średnia dlugosc rozwiazania: 3.0
		\item Średnia ilosc odwiedzonych stanow: 36.8
		\item Średnia maksymalna glebokosc rekursji: 4.0
		\item Średnia ilosc wygenerowanych stanow: 33.0
		\end{itemize}
		Odleglosc rozwiazania: 4
		\begin{itemize}
		\item Średnia dlugosc rozwiazania: 4.0
		\item Średnia ilosc odwiedzonych stanow: 69.04166666666667
		\item Średnia maksymalna glebokosc rekursji: 5.0
		\item Średnia ilosc wygenerowanych stanow: 47.916666666666664
		\end{itemize}
		Odleglosc rozwiazania: 5
		\begin{itemize}
		\item Średnia dlugosc rozwiazania: 5.0
		\item Średnia ilosc odwiedzonych stanow: 168.57407407407408
		\item Średnia maksymalna glebokosc rekursji: 6.0
		\item Średnia ilosc wygenerowanych stanow: 146.35185185185185
		\end{itemize}
		Odleglosc rozwiazania: 6
		\begin{itemize}
			\item Średnia dlugosc rozwiazania: 6.0
			\item Średnia ilosc odwiedzonych stanow: 351.8878504672897
			\item Średnia maksymalna glebokosc rekursji: 7.0
			\item Średnia ilosc wygenerowanych stanow: 270.0841121495327
		\end{itemize}
		
		Kolejnosc przeszukiwania:DLPG
		Odleglosc rozwiazania: 1
		\begin{itemize}
			\item Średnia dlugosc rozwiazania: 1.0
			\item Średnia ilosc odwiedzonych stanow: 4.5
			\item Średnia maksymalna glebokosc rekursji: 2.0
			\item Średnia ilosc wygenerowanych stanow: 7.5
		\end{itemize}
		Odleglosc rozwiazania: 2
		\begin{itemize}
			\item Średnia dlugosc rozwiazania: 2.0
			\item Średnia ilosc odwiedzonych stanow: 17.25
			\item Średnia maksymalna glebokosc rekursji: 3.0
			\item Średnia ilosc wygenerowanych stanow: 19.25
		\end{itemize}
		Odleglosc rozwiazania: 3
		\begin{itemize}
			\item Średnia dlugosc rozwiazania: 3.0
			\item Średnia ilosc odwiedzonych stanow: 38.0
			\item Średnia maksymalna glebokosc rekursji: 4.0
			\item Średnia ilosc wygenerowanych stanow: 35.5
		\end{itemize}
		Odleglosc rozwiazania: 4
		\begin{itemize}
			\item Średnia dlugosc rozwiazania: 4.0
			\item Średnia ilosc odwiedzonych stanow: 70.875
			\item Średnia maksymalna glebokosc rekursji: 5.0
			\item Średnia ilosc wygenerowanych stanow: 51.791666666666664
		\end{itemize}
		Odleglosc rozwiazania: 5
		\begin{itemize}
			\item Średnia dlugosc rozwiazania: 5.0
			\item Średnia ilosc odwiedzonych stanow: 162.37037037037038
			\item Średnia maksymalna glebokosc rekursji: 6.0
			\item Średnia ilosc wygenerowanych stanow: 133.3148148148148
		\end{itemize}
		Odleglosc rozwiazania: 6
		\begin{itemize}
			\item Średnia dlugosc rozwiazania: 6.0
			\item Średnia ilosc odwiedzonych stanow: 362.49532710280374
			\item Średnia maksymalna glebokosc rekursji: 7.0
			\item Średnia ilosc wygenerowanych stanow: 292.7943925233645
		\end{itemize}
		
		Kolejność przeszukiwania:PGDL
		Odleglosc rozwiazania: 1
		\begin{itemize}
			\item Średnia dlugosc rozwiazania: 1.0
			\item Średnia ilosc odwiedzonych stanow: 3.5
			\item Średnia maksymalna glebokosc rekursji: 2.0
			\item Średnia ilosc wygenerowanych stanow: 5.5
		\end{itemize}
		Odleglosc rozwiazania: 2
		\begin{itemize}
			\item Średnia dlugosc rozwiazania: 2.0
			\item Średnia ilosc odwiedzonych stanow: 18.5
			\item Średnia maksymalna glebokosc rekursji: 3.0
			\item Średnia ilosc wygenerowanych stanow: 22.25
		\end{itemize}
		Odleglosc rozwiazania: 3
		\begin{itemize}
			\item Średnia dlugosc rozwiazania: 3.0
			\item Średnia ilosc odwiedzonych stanow: 34.0
			\item Średnia maksymalna glebokosc rekursji: 4.0
			\item Średnia ilosc wygenerowanych stanow: 29.0
		\end{itemize}
		Odleglosc rozwiazania: 4
		\begin{itemize}
			\item Średnia dlugosc rozwiazania: 4.0
			\item Średnia ilosc odwiedzonych stanow: 75.91666666666667
			\item Średnia maksymalna glebokosc rekursji: 5.0
			\item Średnia ilosc wygenerowanych stanow: 62.5
		\end{itemize}
		Odleglosc rozwiazania: 5
		\begin{itemize}
			\item Średnia dlugosc rozwiazania: 5.0
			\item Średnia ilosc odwiedzonych stanow: 161.35185185185185
			\item Średnia maksymalna glebokosc rekursji: 6.0
			\item Średnia ilosc wygenerowanych stanow: 131.14814814814815
		\end{itemize}
		Odleglosc rozwiazania: 6
		\begin{itemize}
			\item Średnia dlugosc rozwiazania: 6.0
			\item Średnia ilosc odwiedzonych stanow: 372.1869158878505
			\item Średnia maksymalna glebokosc rekursji: 7.0
			\item Średnia ilosc wygenerowanych stanow: 313.41121495327104
		\end{itemize}
		
		Kolejnosc przeszukiwania:DGLP
		Odleglosc rozwiazania: 1
		\begin{itemize}
			\item Średnia dlugosc rozwiazania: 1.0
			\item Średnia ilosc odwiedzonych stanow: 4.0
			\item Średnia maksymalna glebokosc rekursji: 2.0
			\item Średnia ilosc wygenerowanych stanow: 6.0
		\end{itemize}
		Odleglosc rozwiazania: 2
		\begin{itemize}
			\item Średnia dlugosc rozwiazania: 2.0
			\item Średnia ilosc odwiedzonych stanow: 16.25
			\item Średnia maksymalna glebokosc rekursji: 3.0
			\item Średnia ilosc wygenerowanych stanow: 16.75
		\end{itemize}
		Odleglosc rozwiazania: 3
		\begin{itemize}
			\item Średnia dlugosc rozwiazania: 3.0
			\item Średnia ilosc odwiedzonych stanow: 38.0
			\item Średnia maksymalna glebokosc rekursji: 4.0
			\item Średnia ilosc wygenerowanych stanow: 36.6
		\end{itemize}
		Odleglosc rozwiazania: 4
		\begin{itemize}
			\item Średnia dlugosc rozwiazania: 4.0
			\item Średnia ilosc odwiedzonych stanow: 79.33333333333333
			\item Średnia maksymalna glebokosc rekursji: 5.0
			\item Średnia ilosc wygenerowanych stanow: 69.79166666666667
		\end{itemize}
		Odleglosc rozwiazania: 5
		\begin{itemize}
			\item Średnia dlugosc rozwiazania: 5.0
			\item Średnia ilosc odwiedzonych stanow: 158.5185185185185
			\item Średnia maksymalna glebokosc rekursji: 6.0
			\item Średnia ilosc wygenerowanych stanow: 124.94444444444444
		\end{itemize}
		Odleglosc rozwiazania: 6
		\begin{itemize}
			\item Średnia dlugosc rozwiazania: 6.0
			\item Średnia ilosc odwiedzonych stanow: 352.21495327102804
			\item Średnia maksymalna glebokosc rekursji: 7.0
			\item Średnia ilosc wygenerowanych stanow: 271.3738317757009
		\end{itemize}
		
	\section{Wnioski}
	
	
	
	\begin{thebibliography}{0}
	  \bibitem. http://en.wikipedia.org/wiki/Depth-first\_search
	  \bibitem. http://en.wikipedia.org/wiki/Iterative\_deepening\_depth-first\_search   
	\end{thebibliography}
\end{document}
